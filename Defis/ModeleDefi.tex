\documentclass[11pt]{article}
\usepackage{a4wide}
\usepackage{times}
\usepackage[french]{babel}
\usepackage[T1]{fontenc} 
\usepackage[utf8]{inputenc}
\usepackage{url}

\usepackage{eurosym}
\usepackage{amssymb}
\usepackage{xcolor}
\newcommand{\mynote}[3][black]{\textcolor{#1}{\fbox{\bfseries\sffamily\scriptsize{#2}}
{\small$\blacktriangleright$\textsf{\emph{#3}}$\blacktriangleleft$}}}

\newcommand{\gpl}[0]{génie de la programmation et du logiciel}
\newcommand{\eg}[0]{\emph{e.g.},~}
\newcommand{\ie}[0]{\emph{i.e.},~}
\newcommand{\etal}[0]{\emph{et al.}~}
\newcommand{\wrt}[0]{\emph{w.r.t.}~}
\newcommand{\cf}[0]{cf.~}



\title{Titre du défi}
\author{Auteurs\\
}
\date{}
\begin{document}
\maketitle

\thispagestyle{empty}
\section{Contenu}
Le Génie de la Programmation et du Logiciel est au c{\oe}ur de l'activité
informatique. Les concepts, méthodes et les outils de conception et de
validation de logiciels constituent les éléments manipulés par les
informaticiens pour maîtriser et automatiser les problèmes qui leur sont
soumis. 
\\
\\

\textbf{Attention pas de numéro de page. Nous intégrerons tous les défis dans un seul document.}

\end{document}
